\documentclass[a4paper]{article}

\usepackage[margin=2cm]{geometry}

\usepackage[T1]{fontenc}
\usepackage[utf8]{inputenc}

\usepackage[unicode,
            colorlinks=true,
            breaklinks,
            urlcolor=cyan, 
            linkcolor=blue, 
            citecolor=green,
            pdftitle={List of some publications related to Combinatorial Reconfiguration}
            ]{hyperref}

\title{List of some publications related to Combinatorial Reconfiguration}
\date{April 12, 2022}

\usepackage[backend=biber, safeinputenc, style=numeric, sorting=ynt, maxbibnames=20, labeldateparts=true, defernumbers=true]{biblatex}

% Note at the end, see https://tex.stackexchange.com/a/139010
\DeclareSourcemap{
	\maps[datatype=bibtex]{
		\map{
			\step[fieldsource=note, final]
			\step[fieldset=addendum, origfieldval, final]
			\step[fieldset=note, null]
		}
	}
}

\addbibresource{core.bib}

% Sort all citations from one year automatically under a heading with the year number
% See https://tex.stackexchange.com/a/472714
\usepackage{xparse}

% from moewe’s answer at https://tex.stackexchange.com/a/434393/105447

\ExplSyntaxOn
\seq_new:N \g__blxbibbyyear_yearlist_seq

\cs_new:Npn \__blxbibbyyear_seq_gput_right_once:Nn #1 #2
{
	\seq_if_in:NnF #1 {#2}
	{ \seq_gput_right:Nn #1 {#2} }
}

\cs_generate_variant:Nn \__blxbibbyyear_seq_gput_right_once:Nn { NV, Nx }


\prg_new_conditional:Nnn \blx_field_if_undef:n { p, T, F , TF }
{
	\use:c { iffieldundef } { #1 } { \prg_return_true: } { \prg_return_false: }
}

% unfortunately, \iffieldint is not expandable, so no p version for us, boo
\prg_new_protected_conditional:Nnn \blx_field_if_int:n { T, F , TF }
{
	\iffieldint { #1 } { \prg_return_true: } { \prg_return_false: }
}

\AtDataInput
{
	\blx_field_if_undef:nF { labeldatesource }
	{
		\blx_field_if_undef:nTF { \thefield{labeldatesource}year }
		{
			\blx_field_if_undef:nF { \thefield{labeldatesource} }
			{
				\blx_field_if_int:nT { \thefield{labeldatesource} }
				{
					\__blxbibbyyear_seq_gput_right_once:Nx \g__blxbibbyyear_yearlist_seq
					{ \thefield{\thefield{labeldatesource}} }
				}
			}
		}
		{
			\__blxbibbyyear_seq_gput_right_once:Nx \g__blxbibbyyear_yearlist_seq
			{ \thefield{\thefield{labeldatesource}year} }
		}
	}
}

\cs_new:Npn \blxbibbyyear_seq_sort_bydirection:NN #1 #2
{
	\seq_sort:Nn #2
	{
		\int_compare:nNnTF { ##1 } #1 { ##2 }
		{ \sort_return_swapped: }
		{ \sort_return_same: }
	}
}

\cs_new:Nn \blxbibbyyear_seq_sort_descending:N
{
	\blxbibbyyear_seq_sort_bydirection:NN < #1
}

\cs_new_nopar:Npn \blxbibbyyear_print_yearbib:nn #1 #2
{
	\defbibcheck{thisyear}
	{
		\blx_field_if_int:nTF { labelyear }
		{
			\int_compare:nNnF { \thefield{labelyear} } = { #1 }
			{ \skipentry }
		}
		{ \skipentry }
	}
	\printbibliography[heading=bibintoc, title=#1, check=thisyear, #2]
}

\DeclareDocumentCommand{\printbibbyyear}{O{}}
{
	\blxbibbyyear_seq_sort_descending:N \g__blxbibbyyear_yearlist_seq
	\seq_map_inline:Nn \g__blxbibbyyear_yearlist_seq
	{ \blxbibbyyear_print_yearbib:nn {##1} {#1} }
}%
\ExplSyntaxOff

% Audrey’s approach to reverse numbering
% at https://tex.stackexchange.com/a/22770/105447

% Count total number of entries in each refsection
\AtDataInput{%
	\csnumgdef{entrycount:\therefsection}{%
		\csuse{entrycount:\therefsection}+1}}

% Print the labelnumber as the total number of entries in the
% current refsection, minus the actual labelnumber, plus one
\DeclareFieldFormat{labelnumber}{\mkbibdesc{#1}}
\newrobustcmd*{\mkbibdesc}[1]{%
	\number\numexpr\csuse{entrycount:\therefsection}+1-#1\relax}

% Change title of TOC
\renewcommand{\contentsname}{Publications categorized by year}

\begin{document}
	\nocite{*}
	
	\maketitle
	
	\begin{abstract}
		For the last few decades, a collection of problems known as \href{https://en.wikipedia.org/wiki/Reconfiguration}{Combinatorial Reconfiguration} has attracted the attention of several researchers all over the world and become a widely applicable young research area (see the surveys~\cite{books/cu/p/Heuvel13,journals/algorithms/Nishimura18,conf/seiccgtc/MynhardtN19}).
		In a reconfiguration setting, \textit{feasible} solutions of a \textit{source} computational problem (such as \textsc{Satisfiability}, \textsc{Independent Set}, \textsc{Dominating Set}, and so on) are given, along with a \textit{reconfiguration rule} that describes how to transform/reconfigure one feasible solution while keeping its feasibility.
		As an example, think of the well-known Rubik's cube puzzle where each state of the cube corresponds to a feasible solution, and the rule is to rotate a face by 90, 180, or 270 degrees.
		This setting naturally defines a so-called \textit{reconfiguration graph} (or \textit{solution graph}) whose nodes are feasible solutions and two nodes are \textit{adjacent} if one is obtained from the other by applying the given rule exactly once.
		One may ask whether there is a \textit{(shortest) path} between two nodes, whether \textit{any} two nodes can be connected by a path, whether the \textit{largest distance} between two nodes is bounded, and so on.
		
		This \textit{reconfiguration} framework has been extremely useful in modeling real-world \textit{dynamic} situations where a desired (old) configuration (of a network, a system, a database, etc.) changes over time (everything does) and needs to be ``reconfigured'' (maybe to a new and better one) without losing certain properties (such as its connectivity, security, etc.).
		In this document, we provide a non-exhaustive list of publications related to this research area.
	\end{abstract}
	
	\clearpage
	
	\tableofcontents
	
	\clearpage
	
 	\printbibbyyear
\end{document}